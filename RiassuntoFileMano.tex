\documentclass[12pt,a4paper]{article}
\usepackage[utf8]{inputenc}
\usepackage[italian]{babel}
\usepackage{amsmath}
\usepackage{underscore}

\title{Competenze Aziendali e Innovazione - Guida Completa}
\author{}
\date{}

\begin{document}

\maketitle


\section{L'Innovazione}

\subsection{Definizione e fonti}
L'innovazione è l'insieme di attività attraverso il quale l'impresa produce e/o attiva nuovi prodotti e nuovi processi produttivi. Può manifestarsi come \textbf{nuovi oggetti} o come \textbf{nuovi modi di fare qualcosa} .

L'innovazione scaturisce da fonti diverse e dai collegamenti che si stabiliscono tra esse. Esempi di fonti includono: imprese, università e centri di ricerca, fondazioni private, enti pubblici .

\subsection{Obiettivo dell'innovazione}
L'obiettivo principale dell'innovazione è migliorare la posizione competitiva dell'impresa per ricavarne un profitto. Tuttavia, l'innovazione comporta un rischio di risultati economici incerti .

\subsection{Scienza, tecnologia e tecnica}
È fondamentale distinguere tra tre concetti correlati ma distinti:

\begin{itemize}
    \item \textbf{Scienza}: è un bene pubblico, caratterizzato da non profittabilità, non segretezza, non-escludibilità e diffusione dei risultati .
    \item \textbf{Tecnologia}: è un bene privato, caratterizzato da segretezza, obiettivi di profitto e protezione brevettale .
    \item \textbf{Tecnica}: rappresenta la materializzazione della tecnologia in progetti, macchine e prodotti concreti .
\end{itemize}

La relazione tra produzione, capitale e lavoro può essere espressa come: \(Y = f(K, L)\), dove Y rappresenta la produzione, K il capitale e L il lavoro .

\subsection{Tipi di innovazione}

\subsubsection{Innovazione radicale}
L'innovazione radicale origina nuove industrie o segmenti di mercato, creando una rottura con i prodotti o processi esistenti. Introduce nuovi prodotti, processi o servizi che creano il mercato. Esempi significativi includono: i cellulari, l'illuminazione elettrica, nuove tecnologie che cambiano completamente il paradigma esistente .

\subsubsection{Innovazione incrementale}
L'innovazione incrementale rappresenta il miglioramento di un processo o servizio rispetto ad uno specifico design dominante. Si tratta di funzionalità aggiunte a prodotti esistenti. Esempi: auto ibride, ADSL, TV a colori .

\subsection{Fasi dell'innovazione}
Il processo di innovazione si articola in quattro fasi fondamentali:

\begin{enumerate}
    \item \textbf{Invenzione}: nuova idea, ma manca l'utilizzo economico, che è il punto fondamentale dell'innovazione. L'invenzione da sola non costituisce innovazione .
    \item \textbf{Progettazione e test}: fase di sviluppo e verifica dell'idea .
    \item \textbf{Realizzazione fisica}: produzione concreta del prodotto o implementazione del processo .
    \item \textbf{Commercializzazione}: introduzione sul mercato e generazione di valore economico .
\end{enumerate}

\subsection{Impresa innovativa}
L'impresa innovativa è quella che riesce a conseguire leadership di mercato nei prodotti e nei servizi offerti ed è in grado di mantenerla nel tempo. È caratterizzata da tre elementi fondamentali :

\begin{itemize}
    \item \textbf{Conoscenza}: l'impresa agevola la conoscenza e la sua diffusione interna .
    \item \textbf{Apprendimento}: l'impresa consente l'apprendimento rapido attraverso formazione, team-working, mentoring e mezzi innovativi .
    \item \textbf{Competenza}: l'impresa raccoglie il risultato dei processi di apprendimento, riassumibile nel concetto di "saper fare le cose" .
\end{itemize}

\section{Diffusione dell'innovazione}

\subsection{Il processo di diffusione}
L'innovazione implica cambiamento e comporta un prezzo. Per essere apprezzata si deve lavorare su due fronti :
\begin{itemize}
    \item \textbf{Comunicazione}: illustrare i vantaggi dell'innovazione
    \item \textbf{Formazione}: fornire aiuto per il cambio di comportamento
\end{itemize}

\subsection{Categorie di adozione secondo Rogers}
Rogers identifica cinque categorie di soggetti che si presentano all'innovazione con atteggiamenti diversi :

\begin{itemize}
    \item \textbf{Innovatori}: esposti a più fonti di innovazione, soggetti ad alto livello di istruzione, disposti a correre rischi .
    \item \textbf{Anticipatori}: alto livello di istruzione, esperienze di successo alle spalle, elevata reputazione sociale .
    \item \textbf{Maggioranza anticipatrice}: soggetti con forte interazione con i pari, posizioni di leadership nella comunità .
    \item \textbf{Maggioranza ritardataria}: soggetti normalmente scettici, tradizionalisti, status economico basso .
    \item \textbf{Ritardatari}: individui normalmente isolati, risorse limitate, processo decisionale lento .
\end{itemize}

\subsection{Fasi dell'adozione}
Il processo di adozione dell'innovazione da parte dell'individuo si articola in cinque fasi :

\begin{enumerate}
    \item \textbf{Consapevolezza}: l'individuo è esposto all'innovazione, senza però avere informazioni approfondite in proposito .
    \item \textbf{Interesse}: l'individuo dispone delle prime informazioni e desidera averne altre .
    \item \textbf{Valutazione}: l'individuo cerca di immaginare l'utilità futura dell'innovazione .
    \item \textbf{Prova}: l'individuo sperimenta l'innovazione in modo limitato .
    \item \textbf{Adozione}: l'individuo decide di applicare completamente l'innovazione .
\end{enumerate}

\subsection{Canali di comunicazione}
Due sono i principali canali di comunicazione dell'innovazione :
\begin{itemize}
    \item \textbf{Canale personale}: più rilevante, in quanto ha il potere di persuadere più forte
    \item \textbf{Attraverso i media}: meno influenti rispetto alla comunicazione personale
\end{itemize}

\subsection{Fattori che influenzano la velocità di diffusione}
Le condizioni esterne o sociali che aumentano o diminuiscono la velocità di diffusione includono :
\begin{itemize}
    \item Le norme sociali dominanti
    \item La presenza di opinion leader
    \item L'esistenza di agenti del mutamento e di aiutanti del cambiamento
\end{itemize}

\section{Pensiero di Schumpeter}

Joseph Schumpeter ha elaborato una teoria dell'innovazione che identifica i seguenti principi fondamentali :

\begin{itemize}
    \item \textbf{L'innovazione è determinante per il mutamento industriale}: rappresenta il motore principale del cambiamento economico .
    \item \textbf{L'innovazione è "fare qualcosa di nuovo"} nel sistema economico .
    \item \textbf{L'innovazione avviene sia nelle grandi che nelle piccole imprese}, anche se è più probabile nelle grandi imprese che hanno a disposizione maggiori risorse .
    \item \textbf{L'innovazione è un evento dai risultati incerti}: l'esito di un'innovazione non è prevedibile .
    \item \textbf{Le innovazioni non sono distribuite uniformemente nel tempo, ma a grappoli}: si alternano periodi di espansione e periodi di recessione .
    \item \textbf{Le imprese giovani tendono ad essere più innovative} .
    \item \textbf{L'innovazione produce un profitto temporaneo} che resta tale finché la reazione delle altre imprese non riequilibra il sistema .
\end{itemize}

\section{Le quattro rivoluzioni industriali}

\subsection{Prima rivoluzione industriale (1784)}
Caratterizzata da: produzione meccanica, strade ferrate, macchine a vapore. Sviluppo di nuove fonti di energia, aumento dell'utilizzo di metalli e minerali, nuovi mezzi di trasporto tra cui navi e treni. Si passa dalla produzione basata sul lavoro umano alla produzione dalle macchine. Nascono le industrie, si ha una migrazione della popolazione dalle campagne alle città e sviluppo di nuove classi sociali. Sviluppo del capitalismo .

\subsection{Seconda rivoluzione industriale (1870)}
Produzione in massa, energia elettrica. Nascita dell'energia elettrica, energia dal petrolio. Forte sviluppo nella chimica. Nascita della produzione di massa in serie. Sviluppo delle telecomunicazioni .

\subsubsection{Taylorismo}
Taylor propone un modello lavorativo organizzato basato su :
\begin{itemize}
    \item Studio scientifico dei migliori metodi di lavoro
    \item Creazione del prototipo di lavoratore adatto alla mansione: selezione del lavoratore ideale
    \item Suddivisione dei compiti tra chi progetta e chi esegue
\end{itemize}

\subsubsection{Fordismo}
Il Fordismo introduce il concetto di "one best way" e si caratterizza per :
\begin{itemize}
    \item Standardizzazione dei modelli
    \item Divisione del lavoro tra lavoratori non specializzati (lavori ripetitivi) e tecnici (operazioni a contenuto concettuale elevato)
    \item Disposizione delle macchine nella sequenza di fabbricazione
    \item Organizzazione delle fasi di montaggio
\end{itemize}

\subsubsection{Toyotismo}
Evoluzione del fordismo, caratterizzato da :
\begin{itemize}
    \item Delocalizzazione e suddivisione della produzione in più siti
    \item Richiesta di prodotti personalizzati
    \item Maggiore richiesta di qualità e numero di controlli
    \item Fare di più eliminando ridondanze e sprechi
\end{itemize}

\subsection{Terza rivoluzione industriale (1969)}
Produzione automatizzata, elettronica e computer. Introduzione di sistemi di calcolo e computer, microprocessori (Intel) e personal computer (PC) con sistema operativo. Introduzione da parte del CERN dell'HTML, dell'internet e del web, DNS .

\subsection{Quarta rivoluzione industriale (oggi)}
Intelligenza artificiale, robotica, big data. In corso: grande digitalizzazione delle macchine da calcolo e loro diffusione. Indirizzo IP in mobilità e diminuzione del costo della capacità di calcolo. Cloud computing. Nascita degli smartphone. Green economy e riduzione dei consumi. Globalizzazione, rapidità, nuove figure professionali. Digital divide tra chi ha accesso alle opportunità e chi ne ha meno. Internet of Things, droni, stampa 3D, robotica .

\subsection{Definizione di rivoluzione industriale}
Una rivoluzione industriale è una profonda e irreversibile trasformazione da parte del sistema economico che raggiunge il sistema produttivo e il sistema sociale .

\section{Tipi di innovazione per area}

L'innovazione può essere classificata in base all'area di applicazione :

\begin{itemize}
    \item \textbf{Innovazione di prodotto}: costruzione di un nuovo oggetto fisico o servizio
    \item \textbf{Innovazione di processo}: miglioramento o modifica del processo produttivo o operativo
    \item \textbf{Innovazione di business model}: nuovi modi di creare, distribuire e catturare valore (esempi: Netflix, car-sharing)
\end{itemize}

\section{Figure coinvolte nell'innovazione}

\subsection{L'imprenditore}
L'imprenditore ricerca l'innovazione, decide di farci un business investendo capitali e risorse, e poi analizza i risultati. Ha un desiderio di affermazione sociale e cerca il profitto, tentando di mantenere e migliorare la sua offerta per battere la concorrenza .

L'imprenditore innovativo deve :
\begin{itemize}
    \item Essere disposto a rischiare
    \item Pensare fuori dagli schemi (pensiero laterale)
    \item Generare idee
    \item Non avere preconcetti su giusto/sbagliato
    \item Cercare di vedere quello che non c'è
\end{itemize}

\subsection{Il valore}
Il valore rappresenta la somma che i compratori sono disposti a pagare per il prodotto di una certa azienda; corrisponde ai ricavi dell'azienda .

Per aumentare il valore si può :
\begin{itemize}
    \item \textbf{Praticare prezzi più alti} dei concorrenti
    \item \textbf{Operare a costi minori} dei concorrenti
    \item \textbf{Aumentare la willingness to pay} (volontà di pagare per quel prodotto) del cliente, andando incontro ai suoi gusti, il che permette di poter praticare prezzi più alti
    \item \textbf{Ottimizzare i processi}: diminuire i costi operativi, usare il capitale con efficienza
\end{itemize}

\section{Catena del valore e vantaggio competitivo}

\subsection{La catena del valore}
La catena del valore è la sequenza di attività eseguite da un'impresa per progettare, vendere e supportare i propri prodotti. Il valore è parte di un \textbf{sistema del valore}, che rappresenta la relazione tra catene del valore diverse (fornitori, distributori, clienti) .

\subsubsection{Attività primarie}
Le attività primarie contribuiscono alla creazione fisica, alla vendita e alle attività post-vendita :

\begin{itemize}
    \item \textbf{Logistica in entrata}: gestione materiali in arrivo dai fornitori
    \item \textbf{Operazioni}: trasformano l'input nella forma finale del prodotto (impianti efficienti, tecnologia necessaria)
    \item \textbf{Logistica in uscita}: stoccaggio del materiale pronto, spedizione in lotti (esempio: Amazon)
    \item \textbf{Marketing e vendite}: definisce come vendere e raggiungere i clienti
    \item \textbf{Servizi post-vendita}: assistenza e supporto alla clientela
\end{itemize}

\subsubsection{Attività di supporto}
Le attività di supporto aggiungono valore di per sé o in funzione della loro relazione con le attività primarie :

\begin{itemize}
    \item \textbf{Acquisti}: trovare materie prime, stabilire legami con i fornitori
    \item \textbf{Sviluppo tecnologico}: innovazione dell'azienda
    \item \textbf{Risorse umane}: patrimonio umano dell'azienda, gestione assunzioni
    \item \textbf{Infrastruttura}: supporto a ogni attività e alle relazioni tra attività
    \item \textbf{Amministrazione}: gestione finanziaria e contabile
\end{itemize}

\subsection{Vantaggio competitivo}
Il vantaggio competitivo viene ottenuto tramite la differenziazione dai concorrenti. Può essere raggiunto attraverso :
\begin{itemize}
    \item Essere migliori
    \item Essere diversi
    \item Praticare prezzi più bassi / avere costi più bassi
\end{itemize}

Un vero vantaggio competitivo richiede tre condizioni :
\begin{enumerate}
    \item Si crea valore per i clienti
    \item Si è capaci di trattenere il valore all'interno dell'impresa
    \item Si è trovato un mezzo per sconfiggere la concorrenza
\end{enumerate}

\section{Strategia}

\subsection{Definizione}
La strategia è il mezzo per creare valore essendo differenti. Indica dove l'impresa deve andare, non il processo per arrivarci .

Ogni strategia deve avere :
\begin{itemize}
    \item Una \textbf{value proposition} identificabile: quali clienti servire, quali bisogni soddisfare e che prezzo praticare
    \item Una \textbf{value chain} specifica per quel business
    \item Una \textbf{scelta} di cosa fare di diverso rispetto ai concorrenti
\end{itemize}

L'obiettivo è conseguire e mantenere un vantaggio competitivo .

\subsection{Value proposition}
La value proposition deve rispondere a tre domande fondamentali :

\begin{itemize}
    \item \textbf{Quali clienti?} Individuare all'interno dell'industry a quali segmenti si intende indirizzare la proposta
    \item \textbf{Quali bisogni?} Comprendere e soddisfare i bisogni di un certo mix di clienti
    \item \textbf{Quale prezzo relativo?} Scegliere se praticare un prezzo alto o basso rispetto alle alternative
\end{itemize}

Se si cerca di servire gli stessi clienti, soddisfare gli stessi bisogni e allo stesso prezzo dei concorrenti, non si ha una strategia !

\subsection{Trade-off}
Trade-off significa effettuare delle scelte: non si possono soddisfare le esigenze di tutti i clienti .

Esempi di trade-off :
\begin{itemize}
    \item Scelta di caratteristiche di prodotto incompatibili
    \item Configurazione di attività incompatibile
    \item Immagine incompatibile
\end{itemize}

La strategia implica scegliere cosa non fare .

\subsection{Quando cambiare strategia}
È necessario cambiare la strategia quando :
\begin{itemize}
    \item I clienti richiedono un cambiamento significativo
    \item La comparsa di innovazioni rende privi di valore i trade-off fondamentali su cui si basa la strategia
    \item Una nuova tecnologia non essenziale annulla totalmente il valore corrente
\end{itemize}

\section{Business model e modello Canvas}

\subsection{Definizione di business model}
Un business model descrive la logica in base alla quale un'organizzazione crea, distribuisce e cattura valore. Si tratta di un insieme di elementi di ontologia aziendale .

Le aree principali da descrivere sono :
\begin{itemize}
    \item \textbf{Prodotto}: capire che cosa si propone di fare l'azienda
    \item \textbf{Clienti}: chi sono i consumatori dell'impresa e come costruire una relazione duratura con loro
    \item \textbf{Infrastruttura}: come vengono gestiti i problemi logistici e infrastrutturali
    \item \textbf{Finanza}: stabilire il modello per ottenere il denaro, come strutturare i costi e rendere il modello sostenibile nel tempo
\end{itemize}

\subsection{Il modello Canvas}
Il Canvas è un foglio stampato in grande, suddiviso in 9 parti che corrispondono a 9 elementi dell'ontologia aziendale. È paragonabile ad una mappa concettuale per categorizzare le idee .

\subsubsection{1. Segmenti di clientela}
Definisce i gruppi di persone o le organizzazioni che l'impresa vuole servire. Gli appartenenti a uno stesso segmento hanno :
\begin{itemize}
    \item Una simile propensione alla spesa
    \item Utilizzano gli stessi canali distributivi
    \item Hanno simili bisogni
    \item Hanno simili disponibilità economiche
\end{itemize}

Il \textbf{targeting} consiste nel valutare ciascun segmento e decidere su quali operare (esempio: mercato di massa vs mercato di nicchia) .

\subsubsection{2. Valore offerto}
La value proposition descrive l'insieme dei prodotti e dei servizi che creano valore per uno specifico segmento di clientela. Si tratta di scegliere se puntare su :
\begin{itemize}
    \item Una proposta \textbf{qualitativa}: focalizzata su innovazione, qualità, marchio
    \item Una proposta caratterizzata da \textbf{prezzi competitivi}
\end{itemize}

Si cerca di capire cosa cercano i clienti del segmento scelto .

\subsubsection{3. Canali di comunicazione, distribuzione e vendita}
I canali servono per informare i clienti sui prodotti e servizi offerti, dare indicazioni su dove acquistarli, e seguire il cliente nella fase post-vendita .

Principali canali :
\begin{itemize}
    \item Forze di vendita dirette (direttamente in contatto col cliente)
    \item Vendite via web
    \item Vendite tramite negozi propri
    \item Vendite attraverso partner e grossisti
\end{itemize}

\subsubsection{4. Relazioni con i clienti}
Insieme di tutti i rapporti instaurati con un certo segmento di clientela, dal momento del primo contatto fino alla terminazione del business .

Esempi: assistenza personalizzata, self-service, servizi automatizzati, community .

\subsubsection{5. Flussi di ricavo}
Descrive i flussi di cassa generati da ogni segmento di clientela. Uno dei punti più delicati corrisponde alla scelta della politica di prezzo più adatta .

Meccanismi di pricing :
\begin{itemize}
    \item \textbf{Prezzo fisso}: listino, in base alle caratteristiche, in base ai volumi
    \item \textbf{Prezzo dinamico}: trattativa, aste, determinazione in base alla domanda
\end{itemize}

\subsubsection{6. Risorse chiave}
Descrive i beni e servizi più importanti per realizzare il business model. Possono essere fisiche, intellettuali, umane, finanziarie .

\subsubsection{7. Attività chiave}
Descrive le cose più importanti che devono essere fatte per realizzare il business model .

Esempi: produzione, problem solving, piattaforma/network .

\subsubsection{8. Partnership chiave}
Descrive la rete dei fornitori e dei partner che permettono la realizzazione del business model .

Le partnership sono realizzate per :
\begin{itemize}
    \item Ridurre i costi ottimizzando le risorse
    \item Ridurre i rischi legati all'incertezza
    \item Acquistare risorse/attività particolari senza doverle sviluppare internamente
\end{itemize}

\subsubsection{9. Struttura dei costi}
Descrive i costi sostenuti per rendere operativo il business model .

\subsection{Canvas - Profilo del cliente e Value Map}
Il Canvas presenta due lati complementari[ _file:2]:

\begin{itemize}
    \item \textbf{Profilo del cliente}: analizza attività, vantaggi e desideri per creare un prodotto/servizio con una proposta di valore interessante[ _file:2].
    \item \textbf{Value Map}: dedicata ai prodotti/servizi offerti. Viene creata una lista dei prodotti/servizi offerti, concentrandosi su come generare vantaggi per il cliente[ _file:2].
\end{itemize}

\subsubsection{Vantaggi del cliente}
Tipi di vantaggi che i clienti cercano[ _file:2]:
\begin{itemize}
    \item Vantaggi attesi
    \item Vantaggi richiesti
    \item Vantaggi desiderati
    \item Vantaggi inaspettati
    \item Vantaggi funzionali
    \item Vantaggi sociali
\end{itemize}

Come creare vantaggi[ _file:2]:
\begin{itemize}
    \item Offrire nuove caratteristiche
    \item Creare conseguenze sociali positive
    \item Aiutare l'adozione di cose nuove
    \item Consentire risparmi
\end{itemize}

\subsubsection{Disagi del cliente}
Tipi di disagio[ _file:2]:
\begin{itemize}
    \item Disagi fisici e intangibili
    \item Ostacoli
    \item Rischi indesiderati
\end{itemize}

Tipi di attività del cliente[ _file:2]:
\begin{itemize}
    \item Attività sociali
    \item Attività personali/emotive
    \item Attività digitali e finanziarie
\end{itemize}

Tipi di soluzione ai disagi[ _file:2]:
\begin{itemize}
    \item Eliminare rischi
    \item Eliminare fonti di frustrazione
    \item Affrontare disagi sociali
    \item Creare risparmi
\end{itemize}

\subsubsection{Fit}
Il \textbf{fit} si ottiene quando i clienti sono interessati alla value proposition: quando l'offerta risponde realmente ai bisogni identificati[ _file:2].

\section{Business Plan}

\subsection{Definizione}
Il business plan è un documento scritto che descrive business, scopi, obiettivi, strategie, mercato e previsioni finanziarie. Un buon business plan è necessario sia per ricercare investitori sia per valutare nuove opportunità interne[ _file:2].

\subsection{Struttura del business plan}
Il business plan è diviso in due parti principali[ _file:2]:

\subsubsection{Parte 1: Illustrazione e sintesi esecutiva}
Include[ _file:2]:
\begin{itemize}
    \item Scopo dell'iniziativa
    \item Prodotti e servizi offerti
    \item Mercati target
    \item Vantaggi competitivi
    \item Descrizione dei processi
\end{itemize}

\subsubsection{Parte 2: Finanza e marketing}
Comprende[ _file:2]:
\begin{itemize}
    \item Analisi dei clienti
    \item Analisi dei concorrenti
    \item Processo produttivo
    \item Piano d'investimento
    \item Vantaggi per l'investitore
    \item Piano economico dettagliato:
    \begin{itemize}
        \item Ricavi
        \item Costi
        \item Risultato economico
        \item Cash flow
        \item Situazione patrimoniale
        \item Capitale necessario
        \item Fabbisogno finanziario
    \end{itemize}
\end{itemize}

\section{Azienda e impresa}

\subsection{Definizione di azienda}
Un'azienda è un'organizzazione di beni economici materiali, immateriali e persone, che punta al soddisfacimento dei bisogni[ _file:2].

\subsection{Classificazione per settore}
Le aziende possono essere classificate in[ _file:2]:
\begin{itemize}
    \item \textbf{Primo settore}: aziende orientate al profitto
    \item \textbf{Secondo settore}: Pubblica Amministrazione, università, trasporti pubblici
    \item \textbf{Terzo settore}: organizzazioni no profit
\end{itemize}

\subsection{Classificazione giuridica}

\subsubsection{Imprese individuali}
Gestite da un singolo imprenditore con partita IVA[ _file:2].

\subsubsection{Società di persone}
Attività portata avanti dai soci dove il fattore personale è centrale[ _file:2]:
\begin{itemize}
    \item Società semplice (SS)
    \item Società in nome collettivo (SNC)
    \item Società in accomandita semplice (SAS)
\end{itemize}

\subsubsection{Società di capitali}
I soci investono quote di capitale e dividono i profitti secondo percentuali prestabilite, con capitale sociale definito[ _file:2]:
\begin{itemize}
    \item Società a responsabilità limitata (SRL)
    \item Società per azioni (SPA)
\end{itemize}

\subsubsection{Società cooperative}
Orientate al soddisfacimento dei bisogni dei soci, con utili da distribuire tramite beni, servizi o opportunità di lavoro[ _file:2].

\subsection{Classificazione per dimensioni}
Secondo le normative europee[ _file:2]:

\begin{itemize}
    \item \textbf{Micro imprese}: meno di 10 persone, fatturato inferiore a 2 milioni di euro
    \item \textbf{Piccole imprese}: meno di 50 dipendenti, fatturato o bilancio fino a 10 milioni di euro
    \item \textbf{Medie imprese}: fino a 250 dipendenti, fatturato fino a 50 milioni o bilancio fino a 43 milioni di euro
    \item \textbf{Grandi imprese}: oltre le soglie sopra indicate
\end{itemize}

\section{Organizzazione aziendale}

\subsection{Definizione}
Un'organizzazione è un aggregato di persone e risorse che opera per un obiettivo comune. L'organizzazione in ambito aziendale nasce nell'ambito industriale dal 1700 e indica le modalità di divisione e coordinamento del lavoro[ _file:2].

\subsection{Struttura organizzativa}
L'organizzazione si articola su tre dimensioni[ _file:2]:

\begin{itemize}
    \item \textbf{Soggetti}: individui, categorie professionali, attori organizzativi
    \item \textbf{Ambiente}: mercati, tecnologia, istituzioni sociali, convenzioni, gerarchie
    \item \textbf{Relazioni sociali}: regole, rapporti interpersonali, contratti
\end{itemize}

\subsection{Funzioni aziendali}
Le funzioni aziendali sono gruppi di attività omogenee e specializzate. Variano in base al tipo di business, alle dimensioni aziendali e alla struttura giuridica[ _file:2].

\section{Organigrammi e gerarchie}

\subsection{Definizione di organigramma}
Gli organigrammi rappresentano graficamente la divisione orizzontale (per funzioni) e verticale (gerarchica) dell'azienda[ _file:2].

\subsection{Organi di linea e di staff}
Si distinguono[ _file:2]:
\begin{itemize}
    \item \textbf{Organi di linea}: responsabili diretti degli obiettivi quantitativi e operativi
    \item \textbf{Organi di staff}: svolgono funzioni di studio, controllo e supporto (esempi: contabilità, risorse umane, qualità)
\end{itemize}

\subsection{Forme di struttura organizzativa}

\subsubsection{Supervisione diretta}
Tipica delle piccole aziende, permette rapido adattamento alle situazioni[ _file:2].

\subsubsection{Suddivisione per funzioni}
Assegnazione chiara di mansioni, rapporti gerarchici e responsabilità[ _file:2].

\subsubsection{Struttura funzionale}
I collaboratori ricevono istruzioni dal superiore diretto, con premi legati al raggiungimento di obiettivi[ _file:2].

\subsubsection{Organizzazione a matrice}
Focus su prodotto o mercato con la presenza di product manager o project manager che coordinano risorse provenienti da diverse funzioni[ _file:2].

\subsubsection{Organizzazione per divisioni}
Raggruppamento per prodotto o area geografica, con top management centrale che coordina le divisioni[ _file:2].

\subsubsection{Organizzazione a rete}
Collaborazione tra più aziende autonome che mantengono la propria indipendenza[ _file:2].

\section{Operations e processi aziendali}

\subsection{Definizione di processo aziendale}
Il processo aziendale è l'insieme strutturato e ripetitivo di attività che creano valore, trasformando risorse (input) in prodotti o servizi (output) per clienti interni o esterni[ _file:2].

\subsection{Tipologie di processi}

\subsubsection{Processi primari}
Creano direttamente valore per il cliente finale[ _file:2].

\subsubsection{Processi di supporto}
Necessari per il funzionamento aziendale ma non generano direttamente valore percepito dal cliente[ _file:2].

\subsubsection{Processi manageriali}
Includono pianificazione strategica, controllo direzionale e controllo operativo[ _file:2].

\subsection{Misurazione dei processi}
I processi devono essere misurabili e monitorabili tramite indicatori di prestazione (KPI - Key Performance Indicators). Gli standard ISO (International Organization for Standardization) forniscono linee guida per la gestione dei processi[ _file:2].

\subsection{Il progetto}
Un progetto è una sequenza di azioni e risorse finalizzate a un obiettivo specifico, tempificato e con limiti di costo definiti[ _file:2].

Caratteristiche del progetto[ _file:2]:
\begin{itemize}
    \item Obiettivo SMART (Specific, Measurable, Achievable, Relevant, Time-bound)
    \item Scadenza precisa
    \item Risorse censite e coordinate
\end{itemize}

\subsubsection{Documentazione di progetto}
Include[ _file:2]:
\begin{itemize}
    \item Diagrammi di Gantt: rappresentazione temporale delle attività
    \item Piano dei lavori: sequenza e dipendenze delle attività
    \item Gestione e aggiornamento dei progressi
    \item Piano dei costi
    \item Individuazione delle attività critiche (critical path)
\end{itemize}

\textbf{EPM} (Enterprise Project Management): metodologia integrata per la gestione dei progetti a livello aziendale[ _file:2].

\section{Software di project management}

\subsection{Principali strumenti}

\subsubsection{Microsoft Project}
Gestione di progetti, costi, report di avanzamento. Software desktop tradizionale per pianificazione e controllo progetti[ _file:2].

\subsubsection{Project Server}
Gestione multipla di progetti online, analisi dettagliate, collaborazione in team[ _file:2].

\subsubsection{Jira}
Particolarmente utilizzato nello sviluppo software, gestione di requisiti, task, bug tracking[ _file:2].

\subsubsection{Trello}
Piattaforma basata su metodologia Kanban per lean production, visualizzazione intuitiva del flusso di lavoro[ _file:2].

\subsubsection{Oracle Primavera}
Integrazione con la commessa aziendale, gestione degli avanzamenti economici, adatto a grandi progetti[ _file:2].

\subsection{Modalità di installazione dei software}

\subsubsection{Standalone}
Installazione su singolo PC, backup manuale dei dati[ _file:2].

\subsubsection{File server}
Dati centralizzati su server, accessibili a più utenti in rete locale[ _file:2].

\subsubsection{On premise}
Database centralizzato su server aziendale, accesso multiplo contemporaneo, backup continuo e automatico[ _file:2].

\subsubsection{Web app}
Accesso via browser da qualsiasi dispositivo, gestione centralizzata degli utenti, identity management integrato[ _file:2].

\section{Cloud Computing e Sistemi Informativi}

\subsection{Il Cloud Computing}
La soluzione cloud permette di utilizzare software non più installato localmente ma fornito come servizio da terzi (service provider), tramite sottoscrizione di un contratto[ _file:3].

Caratteristiche principali[ _file:3]:
\begin{itemize}
    \item Il servizio può essere acquistato solo per il tempo ritenuto necessario
    \item Pagamento per le persone che ne necessitano effettivamente
    \item Importante implementazione dell'identity management anche per servizi esterni
    \item Infrastruttura dotata di collegamenti verso l'esterno di qualità (WAN link veloci, ad alta banda)
    \item Possibile integrazione con altre applicazioni aziendali esistenti
\end{itemize}

\subsection{Sistema informativo}
Un sistema informativo è un insieme di elementi interconnessi che raccolgono, ricercano, elaborano, memorizzano e distribuiscono dati e informazioni utili per supportare le attività decisionali e di controllo di un'azienda[ _file:3].

\textbf{Importante}: non è solo software, ma comprende software, hardware e organizzazione[ _file:3].

\subsection{Macroprocessi del sistema informativo}
Un sistema informativo deve essere progettato per svolgere tre macroprocessi fondamentali[ _file:3]:

\begin{enumerate}
    \item \textbf{Acquisizione}: raccolta e immissione dati (esempio: inserimento ordini, registrazione presenze)
    \item \textbf{Trasformazione/elaborazione}: processamento dei dati per generare informazioni
    \item \textbf{Restituzione}: output di informazioni per clienti, fornitori, management
\end{enumerate}

\subsection{Risorse dei sistemi informativi}

\subsubsection{Risorse tecnologiche}
Computer, server, infrastruttura di rete, servizi cloud[ _file:3].

\subsubsection{Risorse organizzative}
Competenze gestionali necessarie per governare le tecnologie[ _file:3].

\subsubsection{Portafoglio di applicazioni}
Software gestionali e di governo dei dati, a supporto delle varie competenze aziendali[ _file:3].

\subsection{Funzioni dei sistemi informativi}
I sistemi informativi moderni[ _file:3]:
\begin{itemize}
    \item Supportano nuovi modelli di business
    \item Aiutano manager e personale ad analizzare problemi complessi
    \item Creano nuovi prodotti e servizi
    \item Nelle imprese moderne sono componente core e motivo di competizione
\end{itemize}

\subsection{Digitalizzazione}
La digitalizzazione delle informazioni aziendali comporta[ _file:3]:
\begin{itemize}
    \item Sicurezza dei dati tramite meccanismi di blocco e protezione
    \item Applicazioni verticali: specifiche per settore (timbrature, fatturazione, prenotazioni)
    \item Applicazioni orizzontali: supporto multifunzione utilizzabile in diversi contesti
\end{itemize}

\subsection{Intranet}
L'intranet aziendale favorisce la condivisione interna delle informazioni, digitalizzando moduli, procedure e comunicazioni[ _file:3].

\section{Impresa digitale}

\subsection{Definizione}
L'impresa digitale usa le tecnologie per progettare e proporre nuovi servizi, migliorare i processi gestionali e aumentare la competitività[ _file:3].

\subsection{Principali sistemi informativi aziendali}

\subsubsection{ERP - Enterprise Resource Planning}
Soluzioni applicative integrate che coprono[ _file:3]:
\begin{itemize}
    \item Contabilità generale e analitica
    \item Finanza e tesoreria
    \item Gestione del rischio
    \item Funzioni di supporto trasversali
\end{itemize}

\subsubsection{CRM - Customer Relationship Management}
Coordinano[ _file:3]:
\begin{itemize}
    \item Marketing e campagne promozionali
    \item Vendite e gestione opportunità
    \item Assistenza post-vendita e customer service
\end{itemize}

\subsubsection{SCM - Supply Chain Management}
Gestiscono[ _file:3]:
\begin{itemize}
    \item Produzione e pianificazione
    \item Costi operativi
    \item Spedizione e logistica
    \item Consegna al cliente finale
\end{itemize}

La sfida principale è evitare disallineamenti e silos informativi tra le diverse aree aziendali[ _file:3].

\subsection{Governance dei sistemi informativi}

\subsubsection{IT Governance}
Gestione delle infrastrutture tecnologiche: hardware, reti, sicurezza informatica[ _file:3].

\subsubsection{IS Governance}
Logiche e strumenti che rendono coerente il sistema informativo aziendale con la strategia complessiva, all'interno della corporate governance[ _file:3].

\subsubsection{Ruoli chiave}
\begin{itemize}
    \item Chief Information Officer (CIO): responsabile strategico IT
    \item IT Professional: specialisti tecnici
    \item Gestione operativa: amministrazione quotidiana dei sistemi
    \item Supporto utenti: help desk e assistenza
\end{itemize}

\subsection{Make or buy}
Decisione strategica su quali funzioni mantenere internamente e quali esternalizzare[ _file:3]:
\begin{itemize}
    \item \textbf{Make}: funzioni strategiche e core si mantengono all'interno
    \item \textbf{Buy}: funzioni di servizio e non strategiche si affidano all'esterno (outsourcing)
\end{itemize}

\section{Valutazione dei sistemi informativi}

\subsection{Criteri di valutazione}
La valutazione riguarda[ _file:3]:
\begin{itemize}
    \item Valore tecnologico: aggiornamento e adeguatezza delle tecnologie
    \item Leve di strategia: allineamento con gli obiettivi aziendali
    \item Vantaggio competitivo: contributo alla competitività dell'impresa
\end{itemize}

È difficile misurare oggettivamente il valore complessivo e la predisposizione alla flessibilità e al cambiamento[ _file:3].

\subsection{Robustezza e flessibilità}

\subsubsection{Flessibilità}
Capacità di adattarsi a cambiamenti organizzativi o aumenti di carico di lavoro[ _file:3].

\subsubsection{Sicurezza}
Riduzione del rischio di perdita dati, resistenza agli attacchi informatici, continuità operativa[ _file:3].

\subsection{Stakeholder della valutazione}
La valutazione dei sistemi informativi interessa[ _file:3]:
\begin{itemize}
    \item Management aziendale
    \item Management dell'IT
    \item Proprietari e azionisti
    \item Imprenditori
    \item Banche e finanziatori
\end{itemize}

I clienti finali sono generalmente interessati solo al risultato finale, non all'infrastruttura IT[ _file:3].

\section{Gestione del cambiamento}

\subsection{Relazione con l'innovazione}
I sistemi informativi sono strettamente connessi all'innovazione e alla ricerca del valore competitivo. La gestione del cambiamento (change management) consiste nell'eseguire azioni che assicurino il raggiungimento dei risultati attesi tramite l'innovazione[ _file:3].

\subsection{Tipi di cambiamento}

\subsubsection{Innovazione radicale}
Stravolge completamente gli assetti esistenti, richiede un cambiamento profondo dell'organizzazione[ _file:3].

\subsubsection{Miglioramento continuo}
Piccoli cambiamenti costanti e incrementali per migliorare progressivamente i processi[ _file:3].

\section{Cloud Computing - Approfondimento}

\subsection{Caratteristiche del Cloud}
Il cloud offre strumenti flessibili e scalabili, con le seguenti caratteristiche[ _file:3]:
\begin{itemize}
    \item Accessibilità on demand
    \item Self-service: l'utente può attivare servizi autonomamente
    \item Risorse condivise (multi-tenancy)
    \item Accesso via Internet da qualsiasi luogo
    \item Pagamenti pay-per-use: si paga solo per ciò che si utilizza
    \item Gestione documenti digitale invece che cartacea
\end{itemize}

\subsection{Tipi di Cloud}

\subsubsection{Public Cloud}
\begin{itemize}
    \item Accessibile a chiunque sottoscriva il servizio
    \item Risorse condivise tra più clienti
    \item Pagamenti basati sul consumo effettivo
    \item Elevata scalabilità
    \item Costi operativi ridotti
\end{itemize}

\subsubsection{Private Cloud}
\begin{itemize}
    \item Dedicato esclusivamente a una singola organizzazione
    \item Sicurezza e privacy elevate
    \item Maggiore controllo sull'infrastruttura
    \item Costi più elevati
\end{itemize}

\subsubsection{Hybrid Cloud}
\begin{itemize}
    \item Combinazione di public e private cloud
    \item Dati sensibili su cloud privato
    \item Applicazioni meno critiche su cloud pubblico
    \item Flessibilità ottimale
\end{itemize}

\subsection{Caratteristiche specifiche del Public Cloud}
\begin{itemize}
    \item Risorse disponibili on-demand
    \item Elasticità: possibilità di aumentare o diminuire rapidamente le risorse
    \item Scalabilità orizzontale e verticale
    \item Nessun limite geografico
    \item Infrastrutture gestite da grandi operatori (big player)
    \item Modello economico pay-per-use
\end{itemize}

\subsection{Categorie di servizi Cloud}

\subsubsection{SaaS - Software as a Service}
Applicazioni software complete pronte all'uso, accessibili via web. L'utente utilizza il software senza doverlo installare o gestire[ _file:3].

Esempi: Gmail, Office 365, Salesforce[ _file:3].

\subsubsection{PaaS - Platform as a Service}
Ambienti virtuali per utilizzare e sviluppare applicazioni. Fornisce piattaforme di sviluppo complete[ _file:3].

Esempi: Google App Engine, Microsoft Azure, Heroku[ _file:3].

\subsubsection{IaaS - Infrastructure as a Service}
Infrastruttura IT completa: spazio di archiviazione, server virtuali, capacità di calcolo[ _file:3].

Esempi: Amazon AWS, Microsoft Azure Infrastructure, Google Cloud Platform[ _file:3].

\subsection{Vantaggi del Cloud}
\begin{itemize}
    \item Maggiore livello di innovazione possibile
    \item Risorse liberate per lo sviluppo del business
    \item Aumento della flessibilità organizzativa
    \item Miglioramento della sicurezza (se ben gestito)
    \item Abilitazione di nuovi modelli di business
    \item Possibilità di passare rapidamente da MVP (Minimum Viable Product) a prodotto di mercato
\end{itemize}

\subsection{Step di adozione del Cloud}

\subsubsection{Cloud Adoption}
Fase iniziale di sperimentazione su singoli processi non critici[ _file:3].

\subsubsection{Cloud Migration}
Migrazione progressiva di sistemi e applicazioni esistenti su piattaforme cloud[ _file:3].

\subsubsection{Cloud Only}
Totale infrastruttura aziendale basata su cloud, con eliminazione dei data center locali[ _file:3].

\textbf{Attenzione}: fondamentale considerare la proprietà e il controllo delle informazioni aziendali[ _file:3].

\subsection{Quadrante Gartner}
Il quadrante Gartner categorizza le aziende fornitrici di servizi cloud in base a due dimensioni[ _file:3]:
\begin{itemize}
    \item \textbf{Visione}: capacità di anticipare le tendenze del mercato
    \item \textbf{Esecuzione}: capacità di realizzare concretamente la visione
\end{itemize}

Quattro categorie[ _file:3]:
\begin{itemize}
    \item Player di nicchia: focalizzati su segmenti specifici
    \item Visionari: idee innovative ma esecuzione limitata
    \item Leader: eccellenza sia in visione che in esecuzione
    \item Challengers: buona esecuzione ma visione limitata
\end{itemize}

\section{Customer Relationship Management (CRM)}

\subsection{Importanza del capitale relazionale}
Le relazioni con la domanda, i clienti e i partner sono fondamentali per creare valore aziendale. Il capitale relazionale va salvaguardato e sviluppato tramite sistemi CRM[ _file:3].

\subsection{Obiettivi del CRM}
\begin{itemize}
    \item Gestire efficacemente le relazioni con i clienti
    \item Fidelizzare i clienti esistenti
    \item Acquisire nuovi clienti
    \item Integrare strumenti di marketing e vendita
    \item Migliorare il servizio al cliente
\end{itemize}

\subsection{Ciclo di vita del cliente}
Il CRM gestisce il cliente attraverso diverse fasi[ _file:3]:

\begin{enumerate}
    \item \textbf{Identificazione e acquisizione}: individuazione di potenziali clienti e primo contatto
    \item \textbf{Primo acquisto e ingresso}: conversione del prospect in cliente
    \item \textbf{Sviluppo}: raccolta dati, conoscenza approfondita, crescita del valore
    \item \textbf{Maturità e stabilità}: cross-selling (vendita di prodotti complementari), up-selling (vendita di prodotti di fascia superiore)
    \item \textbf{Fase discendente}: riduzione degli acquisti, possibile perdita verso la concorrenza
\end{enumerate}

\subsection{Componenti del CRM}

\subsubsection{CRM operativo}
Automazione delle attività operative di marketing, vendita e assistenza clienti. Gestione quotidiana dei contatti[ _file:3].

\subsubsection{CRM analitico}
Analisi dei dati raccolti, supporto alle decisioni strategiche. Business intelligence applicata alle relazioni con i clienti[ _file:3].

Le due funzioni possono coesistere nello stesso strumento integrato[ _file:3].

\section{Macroprocessi di relazione con il cliente}

\subsection{Marketing operativo}
Il marketing operativo si occupa principalmente di creare nuove occasioni di contatto con clienti attuali e potenziali[ _file:4].

Attività principali[ _file:4]:
\begin{itemize}
    \item Gestione dei contatti e dei lead
    \item Gestione delle campagne pubblicitarie
    \item Telemarketing e teleselling
    \item Gestione del contenuto di marketing
    \item Gestione delle campagne promozionali
    \item Automazione dell'utilizzo di posta elettronica, social media e instant messaging
    \item Personalizzazione della comunicazione sul web
\end{itemize}

\subsection{Vendita}
Il processo di vendita comprende[ _file:4]:

\begin{itemize}
    \item Gestione dei contatti commerciali
    \item Gestione delle opportunità di vendita (sales pipeline)
    \item Gestione delle proposte commerciali e preventivi
    \item Gestione dei clienti: raccolta di informazioni rilevanti ai fini di vendita
    \item Configurazione di prodotti/servizi venduti
    \item Gestione dei cataloghi e dei prezzi di listino
    \item Gestione della fase di vendita, comprese le spese
    \item Gestione della vendita tramite web (e-commerce)
\end{itemize}

\subsection{Customer Service}
Il customer service gestisce la relazione con i clienti dopo la vendita[ _file:4].

Può essere di due tipi[ _file:4]:
\begin{itemize}
    \item \textbf{Passivo}: risposta alle esigenze dei clienti (help desk, supporto reattivo)
    \item \textbf{Attivo}: verifica proattiva che la propria offerta soddisfi il cliente (esempio: richiesta di feedback via email come fa Amazon)
\end{itemize}

Attività specifiche[ _file:4]:
\begin{itemize}
    \item Gestione delle richieste di assistenza
    \item Assistenza self-service da web (FAQ, knowledge base)
    \item Gestione degli interventi sul territorio (field service)
    \item Gestione e valorizzazione della knowledge base interna
\end{itemize}

\subsection{Customer Database}
Sistema informativo che raccoglie[ _file:4]:
\begin{itemize}
    \item Dati anagrafici e sociodemografici sui clienti
    \item Prodotti venduti a ciascun cliente
    \item Condizioni di vendita applicate
    \item Tempistica degli acquisti
    \item Marketing: tipologia e quantità di campagne effettuate
\end{itemize}

Permette analisi avanzate e correlazione dei dati per identificare pattern e tendenze[ _file:4].

\subsection{Workflow}
Automazione dei flussi di dati e dei processi organizzativi[ _file:4]:
\begin{itemize}
    \item Autorizzazione automatica degli sconti
    \item Censimento di nuova anagrafica cliente
    \item Gestione dell'esito di trattative
    \item Chiusura opportunità con apertura automatica dell'ordine
    \item Comunicazioni automatiche ai vari attori del processo
\end{itemize}

\section{CRM Analitico}

\subsection{Obiettivi del CRM analitico}
\begin{itemize}
    \item Valorizzare il patrimonio informativo acquisito dal CRM operativo
    \item Prendere decisioni strategiche informate
    \item Attuare azioni innovative per aumentare il numero e la qualità delle relazioni d'impresa
\end{itemize}

\subsection{Funzioni del CRM analitico}

\subsubsection{CRM Warehouse}
Database ottimizzato per la produzione di report avanzati e analisi multidimensionali[ _file:4].

\subsubsection{CRM Intelligence}
Produzione e analisi in tempo reale dei dati per supporto alle decisioni di vendita, marketing e servizio clienti[ _file:4].

\subsubsection{Supporto decisionale}
Analisi predittive, segmentazione avanzata della clientela, identificazione di opportunità di cross-selling e up-selling[ _file:4].

\subsection{Vantaggi dell'introduzione del CRM}
\begin{itemize}
    \item Maggiore soddisfazione dei clienti grazie a servizi personalizzati
    \item Mantenimento completo della storia del cliente
    \item Riduzione dei costi di marketing diretto
    \item Maggiore efficacia delle campagne di marketing
    \item Migliore fidelizzazione e riduzione del churn rate
    \item Aumento del customer lifetime value
\end{itemize}

\subsection{Fattori critici di successo}
Il successo di un'implementazione CRM dipende da[ _file:4]:
\begin{itemize}
    \item Settore di appartenenza e maturità del mercato
    \item Caratteristiche specifiche dei clienti target
    \item Cultura organizzativa dell'azienda
    \item Adeguatezza dei sistemi informativi esistenti
    \item Formazione del personale
    \item Supporto del top management
    \item Change management efficace
\end{itemize}

\subsection{Tipologie di CRM}

\subsubsection{CRM nativi}
Software sviluppati specificatamente per la gestione delle relazioni con i clienti. Sono progettati da zero per questo scopo e offrono funzionalità complete e integrate. Possono essere integrati con sistemi di contabilità e strumenti di mass mailing[ _file:4].

\subsubsection{Estensioni CRM}
Moduli CRM aggiuntivi che si integrano con software gestionali esistenti (tipicamente ERP). Offrono un percorso di adozione più graduale[ _file:4].

\section{Enterprise Resource Planning (ERP)}

\subsection{Definizione e obiettivi}
L'ERP è una soluzione informatica integrata che copre tutte le esigenze informative delle diverse aree aziendali. Risolve il problema della frammentazione dei sottosistemi isolati e delle interruzioni nei flussi operativi[ _file:4].

\subsection{Caratteristiche principali degli ERP}

\subsubsection{Modularità}
Suddivisione in moduli funzionali che possono essere implementati progressivamente in base alle necessità aziendali[ _file:4].

\subsubsection{Integrabilità}
Capacità di integrarsi con sistemi esterni e applicazioni di terze parti tramite API e interfacce standard[ _file:4].

\subsubsection{Parametrizzabilità}
Possibilità di configurazione sulle specificità aziendali senza modificare il codice sorgente[ _file:4].

\subsubsection{Flessibilità e adattabilità}
Capacità di adattarsi ai cambiamenti organizzativi e ai nuovi requisiti di business[ _file:4].

\subsubsection{Accessibilità}
Facilità di estrazione ed esportazione dei dati per analisi e reporting[ _file:4].

\subsubsection{Sicurezza e affidabilità}
Tutela della riservatezza dei dati, controllo degli accessi, audit trail completo[ _file:4].

\subsection{Criticità nell'adozione degli ERP}
\begin{itemize}
    \item \textbf{Investimento significativo}: tecnologico e organizzativo notevole
    \item \textbf{Complessità implementativa}: richiede change management essenziale
    \item \textbf{Tempi lunghi}: l'implementazione può richiedere mesi o anni
    \item \textbf{Necessità di Data Warehouse}: per analisi avanzate dei dati integrati
    \item \textbf{Costi di adozione}: licenze, consulenza, formazione
    \item \textbf{Inefficienza iniziale}: curva di apprendimento del personale
    \item \textbf{Costi di manutenzione}: aggiornamenti e adattamenti continui
    \item \textbf{Costi di transizione}: in caso di cambio software (vendor lock-in)
\end{itemize}

\subsection{Principali moduli ERP}

\subsubsection{Logistica vendite}
Gestione di ordini clienti, consegne, contratti di vendita, pianificazione delle spedizioni[ _file:4].

\subsubsection{Produzione}
Pianificazione della produzione (MRP - Material Requirements Planning), controllo avanzamento, gestione della capacità produttiva[ _file:4].

\subsubsection{Approvvigionamento}
Gestione degli acquisti, gestione delle scorte, movimentazione del magazzino[ _file:4].

\subsubsection{Contabilità}
Contabilità generale, contabilità analitica, bilancio, controlling[ _file:4].

\subsubsection{Risorse umane}
Amministrazione del personale, gestione presenze, formazione, payroll[ _file:4].

\subsubsection{Pianificazione business}
Budget, forecast, pianificazione strategica[ _file:4].

\subsubsection{Gestione cespiti e progetti}
Gestione del capitale fisso (asset management), project management integrato[ _file:4].

\subsection{Tipologie di ERP}

\subsubsection{ERP locale}
Soluzioni specifiche per le esigenze nazionali: legislazione fiscale, contabile, normative locali[ _file:4].

\subsubsection{ERP internazionale}
Gestione multi-lingua e multivaluta, conformità a normative di diversi paesi[ _file:4].

\subsubsection{ERP singola azienda}
Gestione centralizzata di un'unica entità legale[ _file:4].

\subsubsection{ERP multi-azienda}
Gestione di aziende separate mantenendo la loro autonomia contabile[ _file:4].

\subsubsection{ERP per gestione gruppo}
Integrazione tra aziende diverse collegate in un gruppo societario, con consolidamento dei dati[ _file:4].

\subsection{Aspetti tecnici degli ERP}
Gli ERP complessi richiedono[ _file:4]:
\begin{itemize}
    \item Infrastrutture dedicate e personale IT qualificato
    \item Ambiente di produzione separato da ambiente di test
    \item Ambiente di sviluppo per personalizzazioni
    \item Sistemi di alta affidabilità (high availability)
    \item Piani di disaster recovery e business continuity
    \item Partner strategici per aggiornamenti tecnici e supporto continuativo
\end{itemize}

\subsection{Ruoli nel ciclo di vita ERP}

\subsubsection{Casa madre produttrice}
Sviluppa il software, definisce la roadmap di prodotto, rilascia aggiornamenti[ _file:4].

\subsubsection{Implementatore}
Partner certificato che installa e configura il software ERP[ _file:4].

\subsubsection{Consulente direzionale}
Supporta la gestione del cambiamento organizzativo, rivede i processi aziendali[ _file:4].

\subsubsection{Attori specializzati}
Responsabili per[ _file:4]:
\begin{itemize}
    \item Setup iniziale del sistema
    \item Parametrizzazione: configurazione dei parametri
    \item Verticalizzazione: adattamento al settore specifico
    \item Personalizzazione: sviluppi custom
    \item Versioning e aggiornamenti: gestione delle release
\end{itemize}

\subsection{Competenze richieste}
L'implementazione e gestione di un ERP richiede[ _file:4]:
\begin{itemize}
    \item Competenze sistemistiche (infrastruttura IT)
    \item Project management
    \item Sviluppo software e programmazione
    \item Customer care e supporto utenti
    \item Conoscenza dei processi aziendali
    \item Relazioni con istituzioni e normative
\end{itemize}

\section{Supply Chain Management (SCM)}

\subsection{Definizione}
L'SCM è un insieme di soluzioni software che supervisionano il flusso di materiali e informazioni dai fornitori, attraverso la produzione e la distribuzione, fino al cliente finale[ _file:4].

\subsection{Obiettivi del SCM}
\begin{itemize}
    \item Riduzione dei costi di approvvigionamento
    \item Riduzione dei tempi e dei costi di produzione
    \item Riduzione dei costi di fornitura e trasporto
    \item Riduzione degli inventari e delle scorte
    \item Maggiore consapevolezza nella selezione dei fornitori
    \item Aumento della qualità dei prodotti
    \item Miglioramento del servizio al cliente
    \item Sincronizzazione delle attività lungo tutta la catena distributiva
    \item Visibilità end-to-end della supply chain
\end{itemize}

\subsection{Principali processi del SCM}

\subsubsection{Pianificazione}
Equilibrio tra domanda prevista e capacità produttiva. Demand planning e supply planning[ _file:4].

\subsubsection{Approvvigionamento}
Selezione dei fornitori e dei materiali, negoziazione contratti, gestione ordini di acquisto, stoccaggio delle materie prime[ _file:4].

\subsubsection{Trasformazione}
Lavorazione delle materie prime in prodotti finiti, packaging, controllo qualità[ _file:4].

\subsubsection{Distribuzione}
Consegna dei prodotti finiti ai clienti, gestione degli ordini, gestione dei magazzini periferici[ _file:4].

\subsubsection{Rientro}
Gestione dei resi, gestione dei difetti post-vendita, reverse logistics[ _file:4].

\section{Sistemi Informativi Aziendali}

\subsection{Document Management}

\subsubsection{Definizione}
La gestione documentale digitale è un sistema hardware/software per trattare informazioni sia interne che ricevute da soggetti esterni, in formato strutturato e non strutturato. La dematerializzazione trasforma non solo gli oggetti fisici ma anche i processi[ _file:4].

\subsubsection{Funzionalità principali}
\begin{itemize}
    \item Gestione dell'archiviazione digitale sostitutiva (a norma di legge, CAD - Codice dell'Amministrazione Digitale)
    \item Integrazione con altri sistemi aziendali per la gestione dei flussi documentali
    \item Utilizzo di codici a barre e QR code per il tracciamento dei documenti
\end{itemize}

\subsubsection{Caratteristiche tecniche}
\begin{itemize}
    \item Database per repository documentale centralizzato
    \item Interfaccia utente per consultazione e ricerca
    \item Interfacce per acquisizione (scanner, import) e generazione documenti
    \item Ambiente di personalizzazione dei flussi documentali (workflow)
    \item Conservazione a lungo periodo con garanzie legali
    \item Disponibilità per controlli esterni (audit, verifiche fiscali)
\end{itemize}

\subsection{Ambienti Collaborativi}

\subsubsection{Definizione}
Soluzioni per la condivisione di informazioni e la collaborazione tra utenti interni ed esterni all'organizzazione[ _file:4].

\subsubsection{Caratteristiche}
\begin{itemize}
    \item Diffusione della digitalizzazione tramite smartphone (BYOD - Bring Your Own Device)
    \item Sistemi di messaggistica istantanea aziendale
    \item Task assignment e gestione attività
    \item Video meeting e conferenze online
    \item Organizzazione mobile-first degli spazi di lavoro condivisi
    \item Self-service per gli utenti
    \item Sicurezza e controllo degli accessi granulare
    \item Backup automatico e affidabilità dei dati
    \item Legame forte tra azienda e vendor dei servizi collaborativi
\end{itemize}

\section{Decision Support Systems (DSS) e Data Warehouse}

\subsection{Decision Support System}

\subsubsection{Definizione}
Soluzioni software per il supporto alle decisioni aziendali, basate su analisi di dati interni ed esterni, comunicazione, condivisione e monitoraggio delle decisioni[ _file:4].

\subsubsection{Funzionalità}
\begin{itemize}
    \item Analisi what-if e simulazioni
    \item Modellazione predittiva
    \item Supporto alle decisioni strategiche
    \item Dashboard interattive per il management
\end{itemize}

\subsection{Data Warehouse}

\subsubsection{Definizione}
Il Data Warehouse è un magazzino dati che raccoglie informazioni provenienti da diversi sistemi aziendali[ _file:4].

\subsubsection{Caratteristiche}
\begin{itemize}
    \item Normalizzazione dei dati per garantire coerenza
    \item Glossario aziendale per interpretazione uniforme dei dati
    \item Organizzazione in Data Mart tematici per singole aree di business
    \item Analisi multidimensionali (OLAP - Online Analytical Processing)
    \item Utilizzo di cubi di dati per navigazione rapida
\end{itemize}

\subsubsection{Modalità di utilizzo}
\begin{itemize}
    \item \textbf{User-driven}: gli utenti possono analizzare i dati in autonomia con strumenti di self-service BI
    \item \textbf{Report predefiniti}: report standard distribuiti periodicamente
\end{itemize}

\subsection{Analisi dei dati}

\subsubsection{Tecniche di analisi}
\begin{itemize}
    \item \textbf{Reportistica avanzata}: report personalizzati e interattivi
    \item \textbf{Analisi Drill Down}: approfondimento progressivo su risultati aggregati
    \item \textbf{KPI - Key Performance Indicator}: indicatori chiave per il monitoraggio delle performance aziendali
\end{itemize}

Utilizzo crescente di intelligenza artificiale e machine learning per approfondire la conoscenza e comprendere le dinamiche aziendali[ _file:4].

\subsection{Aspetti tecnici del Data Warehouse}
\begin{itemize}
    \item Necessità di database di grandi dimensioni con architetture scalabili
    \item Prestazioni elevate per l'esecuzione di query complesse e aggiornamenti batch
    \item Soluzioni Cloud per Data Warehouse (AWS Redshift, Google BigQuery, Azure Synapse)
    \item Sicurezza dei dati e controllo degli accessi
    \item Recuperabilità dei dati in caso di guasti
    \item Alta accessibilità e disponibilità del servizio
\end{itemize}

\section{Project e Idea Management}

\subsection{Project Management}
Gestione completa dei progetti aziendali attraverso[ _file:4]:
\begin{itemize}
    \item Assegnazione di task ai membri del team
    \item Controllo dell'avanzamento rispetto al piano
    \item Rendicontazione economica e temporale
    \item Valutazione delle performance individuali e di team
    \item Gestione dei rischi e delle issue
    \item Comunicazione con gli stakeholder
\end{itemize}

\subsection{Idea Management}
Gestione creativa delle idee attraverso[ _file:4]:
\begin{itemize}
    \item Sessioni di brainstorming strutturate
    \item Mappe mentali e diagrammi concettuali
    \item Software dedicati per la raccolta e valutazione delle idee
    \item Riunioni in presenza e a distanza
    \item Piattaforme collaborative per l'innovazione
    \item Processi di selezione e prioritizzazione delle idee
\end{itemize}

\section{Firma elettronica, SPID e Conservazione}

\subsection{Tipi di Firma Elettronica}

\subsubsection{Firma elettronica semplice}
Autenticazione tramite user-id e password, livello di sicurezza base[ _file:4].

\subsubsection{Firma elettronica avanzata (FEA)}
Identifica univocamente il firmatario e consente la verifica dell'integrità del documento e di eventuali modifiche dei dati[ _file:4].

\subsubsection{Firma qualificata}
FEA basata su certificato qualificato, realizzata tramite dispositivo sicuro per la creazione della firma, conforme ai regolamenti eIDAS (electronic IDentification Authentication and Signature)[ _file:4].

\subsubsection{Firma digitale}
FEA con crittografia a chiavi asimmetriche, certificato qualificato rilasciato da un'autorità di certificazione accreditata. Ha pieno valore legale[ _file:4].

\subsection{SPID}

\subsubsection{Definizione}
Sistema Pubblico di Identità Digitale con credenziali uniche (user-id e password) per l'accesso sicuro ai servizi online della Pubblica Amministrazione e di soggetti privati aderenti[ _file:4].

\subsubsection{Livelli di sicurezza}
\begin{itemize}
    \item \textbf{Livello 1}: username e password
    \item \textbf{Livello 2}: autenticazione a due fattori con OTP (One Time Password)
    \item \textbf{Livello 3}: autenticazione forte con smartcard o token hardware
\end{itemize}

\subsection{Conservazione Sostitutiva}

\subsubsection{Definizione}
Conservazione digitale dei documenti a norma di legge (CAD - Codice dell'Amministrazione Digitale), necessaria per garantire[ _file:4]:
\begin{itemize}
    \item Autenticità del documento
    \item Integrità: il documento non è stato alterato
    \item Leggibilità nel tempo
    \item Reperibilità: facile ricerca e consultazione
\end{itemize}

\subsubsection{Requisiti tecnici}
\begin{itemize}
    \item Apposizione di marca temporale
    \item Firma digitale del responsabile della conservazione
    \item Sistema di conservazione a norma
    \item Manuale di conservazione
    \item Piano di sicurezza dei dati
\end{itemize}

\subsection{PEC - Posta Elettronica Certificata}

\subsubsection{Caratteristiche}
La PEC è equiparata alla raccomandata con ricevuta di ritorno[ _file:4]:
\begin{itemize}
    \item Garantisce l'avvenuta consegna del messaggio
    \item Fornisce ricevuta di accettazione e di consegna
    \item Certifica data e ora di invio e ricezione
    \item Ha valore legale per le comunicazioni ufficiali
    \item Tracciabilità completa del messaggio
\end{itemize}

\subsection{Fatturazione Elettronica}

\subsubsection{Normativa}
Obbligatoria per la Pubblica Amministrazione dal 2014 e per le imprese private dal 2019 in Italia[ _file:4].

\subsubsection{Funzionamento}
\begin{itemize}
    \item Basata su file XML con formato standard (FatturaPA)
    \item Trasmissione tramite SDI (Sistema di Interscambio)
    \item Ricevute e conservate da provider autorizzati
    \item Controlli automatici di validità formale
    \item Consegna al destinatario tramite PEC o codice destinatario
    \item Conservazione digitale obbligatoria per 10 anni
\end{itemize}

\section{GDPR e Privacy}

\subsection{Definizione}
Regolamento europeo sulla protezione dei dati personali (General Data Protection Regulation), pubblicato nel 2016 e diventato operativo nel maggio 2018[ _file:4].

\subsection{Obiettivi}
\begin{itemize}
    \item Tutela del trattamento dei dati personali dei cittadini europei
    \item Regolamentazione dei comportamenti aziendali
    \item Standardizzazione dei processi di raccolta e gestione dati
    \item Responsabilizzazione delle organizzazioni (accountability)
    \item Trasparenza nei confronti degli interessati
\end{itemize}

\subsection{Ruoli definiti dal GDPR}

\subsubsection{Titolare del trattamento}
Soggetto che decide le finalità e le modalità del trattamento dei dati personali[ _file:4].

\subsubsection{Responsabile del trattamento}
Soggetto che attua il trattamento su incarico del titolare, seguendo le sue istruzioni[ _file:4].

\subsubsection{Destinatario}
Soggetto che riceve i dati personali dal titolare per finalità specifiche[ _file:4].

\subsubsection{Incaricato del trattamento}
Persona fisica che gestisce operativamente i dati sotto la direzione del titolare o del responsabile[ _file:4].

\subsubsection{Interessato}
Persona fisica a cui i dati personali si riferiscono[ _file:4].

\subsubsection{Rappresentante}
Soggetto designato dal titolare per garantire l'operatività del GDPR nell'Unione Europea[ _file:4].

\subsection{Concetti fondamentali del GDPR}

\subsubsection{Data breach}
Violazione di sicurezza che comporta la perdita, la distruzione o l'accesso illecito ai dati personali[ _file:4].

Obblighi in caso di data breach:
\begin{itemize}
    \item Notifica all'autorità di controllo entro 72 ore
    \item Comunicazione agli interessati se il rischio è elevato
    \item Documentazione dell'incidente
\end{itemize}

\subsubsection{Pseudonimizzazione}
Trattamento dei dati in modo tale che non possano più essere attribuiti a un interessato specifico senza informazioni aggiuntive. Include tecniche di cifratura e anonimizzazione[ _file:4].

\subsubsection{Privacy by Design}
Integrazione della protezione dei dati fin dalla progettazione di sistemi, processi e prodotti[ _file:4].

\subsubsection{Privacy by Default}
Applicazione automatica delle impostazioni di privacy più restrittive, senza necessità di intervento dell'utente[ _file:4].

\subsection{Diritti degli interessati}
\begin{itemize}
    \item Diritto di accesso ai propri dati
    \item Diritto di rettifica
    \item Diritto alla cancellazione (diritto all'oblio)
    \item Diritto di limitazione del trattamento
    \item Diritto alla portabilità dei dati
    \item Diritto di opposizione al trattamento
\end{itemize}

\subsection{Sanzioni}
Il GDPR prevede sanzioni amministrative pecuniarie fino a 20 milioni di euro o fino al 4\% del fatturato mondiale annuo, a seconda di quale importo sia maggiore.

\section{Ruoli IT e Certificazioni}

\subsection{Professionalità nell'area IT}

\subsubsection{Software Development}
Figure professionali coinvolte nello sviluppo software[ _file:4]:
\begin{itemize}
    \item \textbf{Analista}: raccolta e analisi dei requisiti
    \item \textbf{Architetto software}: progettazione dell'architettura di sistema
    \item \textbf{Full Stack Developer}: sviluppo completo front-end e back-end
    \item \textbf{Back End Developer}: sviluppo lato server, database, API
    \item \textbf{Front End Developer}: sviluppo interfacce utente
    \item \textbf{Low Code Developer}: sviluppo con piattaforme low-code
    \item \textbf{Low Level Developer}: programmazione a basso livello, sistemi embedded
    \item \textbf{Web Developer}: sviluppo applicazioni web
    \item \textbf{CMS Developer}: sviluppo su Content Management Systems
    \item \textbf{Web Designer}: progettazione grafica e UX/UI
    \item \textbf{Webmaster}: gestione e manutenzione siti web
\end{itemize}

\subsubsection{Data e Business Intelligence}
\begin{itemize}
    \item \textbf{Data Analyst}: analisi dati e reportistica
    \item \textbf{Data Scientist}: modelli predittivi e machine learning
    \item \textbf{BI Developer}: sviluppo soluzioni di Business Intelligence
\end{itemize}

\subsubsection{Infrastructure e Operations}
\begin{itemize}
    \item \textbf{System Engineer}: progettazione infrastrutture IT
    \item \textbf{Network Administrator}: gestione reti e connettività
    \item \textbf{System Administrator}: gestione server e sistemi operativi
    \item \textbf{Database Administrator (DBA)}: gestione database
    \item \textbf{Security Manager}: gestione sicurezza informatica
    \item \textbf{DevOps Engineer}: integrazione sviluppo e operations
\end{itemize}

\subsubsection{Support}
\begin{itemize}
    \item \textbf{End User Support}: assistenza utenti finali
    \item \textbf{Desktop Support}: supporto hardware e software desktop
    \item \textbf{Help Desk}: primo livello di supporto
\end{itemize}

\subsubsection{Management}
\begin{itemize}
    \item \textbf{Project Manager}: gestione progetti IT
    \item \textbf{IT Manager}: responsabile area IT
    \item \textbf{Information Security Manager}: responsabile sicurezza informazioni
    \item \textbf{CIO - Chief Information Officer}: direttore sistemi informativi
\end{itemize}

\subsubsection{Sales e Pre-Sales}
\begin{itemize}
    \item \textbf{Sales}: vendita soluzioni IT
    \item \textbf{Pre-Sales}: supporto tecnico alla vendita
    \item \textbf{Sales Trainer}: formazione reti vendita
    \item \textbf{Speaker}: presentazioni pubbliche e eventi
\end{itemize}

\subsection{Certificazioni di settore}

\subsubsection{ITIL - Information Technology Infrastructure Library}
Framework per il management dei servizi IT[ _file:4].

Cinque volumi principali:
\begin{enumerate}
    \item \textbf{Service Strategy}: strategia del servizio IT
    \item \textbf{Service Design}: progettazione dei servizi
    \item \textbf{Service Transition}: transizione verso nuovi servizi
    \item \textbf{Service Operation}: gestione operativa quotidiana
    \item \textbf{Continual Service Improvement}: miglioramento continuo
\end{enumerate}

\subsubsection{Project Management}
Certificazioni per la gestione progetti[ _file:4]:
\begin{itemize}
    \item \textbf{PMP - Project Management Professional}: certificazione PMI
    \item \textbf{PMBOK - Project Management Body of Knowledge}: standard di riferimento
    \item \textbf{PRINCE2}: metodologia strutturata per il project management
    \item \textbf{Agile/Scrum}: certificazioni per metodologie agili
\end{itemize}

\subsubsection{ISO 27001}
Standard internazionale per il trattamento sicuro delle informazioni e gestione della sicurezza informatica[ _file:4].

Copre:
\begin{itemize}
    \item Sistema di gestione della sicurezza delle informazioni (ISMS)
    \item Valutazione e gestione dei rischi
    \item Controlli di sicurezza
    \item Audit e conformità
\end{itemize}

\subsubsection{Altre certificazioni rilevanti}
\begin{itemize}
    \item \textbf{CISSP}: Certified Information Systems Security Professional
    \item \textbf{CISA}: Certified Information Systems Auditor
    \item \textbf{AWS/Azure/Google Cloud}: certificazioni cloud specifiche
    \item \textbf{CompTIA}: certificazioni fondamentali IT
    \item \textbf{Cisco}: certificazioni networking (CCNA, CCNP)
    \item \textbf{Microsoft}: certificazioni su tecnologie Microsoft
    \item \textbf{Oracle}: certificazioni database e applicazioni
\end{itemize}

\section{Skills e Competenze}

\subsection{Hard Skills}
Competenze tecnologiche specifiche e misurabili[ _file:4]:
\begin{itemize}
    \item Linguaggi di programmazione
    \item Database e SQL
    \item Sistemi operativi
    \item Networking
    \item Cloud computing
    \item Sicurezza informatica
    \item Metodologie di sviluppo
    \item Strumenti specifici e tecnologie
\end{itemize}

\subsection{Soft Skills}
Competenze trasversali e comportamentali[ _file:4]:
\begin{itemize}
    \item \textbf{Comunicazione}: capacità di esprimere idee chiaramente
    \item \textbf{Problem solving}: risoluzione efficace di problemi complessi
    \item \textbf{Collaborazione}: lavoro efficace in team
    \item \textbf{Creatività}: pensiero innovativo e laterale
    \item \textbf{Curiosità}: desiderio di apprendere continuamente
    \item \textbf{Adattabilità}: flessibilità al cambiamento
    \item \textbf{Time management}: gestione efficace del tempo
    \item \textbf{Leadership}: capacità di guidare e motivare
\end{itemize}

\subsection{Competenze organizzative}
\begin{itemize}
    \item \textbf{Competenze personali}: autonomia, iniziativa, responsabilità
    \item \textbf{Competenze relazionali}: empatia, ascolto attivo, networking
    \item \textbf{Competenze metodologiche}: organizzazione, pianificazione, analisi
\end{itemize}

\subsection{Apprendimento continuo}
Nel settore IT è fondamentale:
\begin{itemize}
    \item Formazione continua sulle nuove tecnologie
    \item Partecipazione a comunità professionali
    \item Sperimentazione e progetti personali
    \item Lettura di documentazione tecnica
    \item Partecipazione a conferenze e eventi
    \item Networking professionale
\end{itemize}

\end{document}

