
\section{La terza rivoluzione industriale}
La terza rivoluzione industriale, iniziata nel 1969, è caratterizzata dalla produzione automatizzata di dispositivi elettronici e computer. Questo periodo si colloca nel secondo dopoguerra, un’epoca di ricostruzione globale, in cui in Europa e America si svilupparono le prime invenzioni informatiche di portata mondiale. Un elemento fondamentale di questo progresso fu la nascita delle prime macchine da calcolo collegate tra loro, segnando l’inizio di una nuova era nella comunicazione e nel trattamento dei dati.

Molti fattori contribuirono ad accelerare questo sviluppo tecnologico. Tra questi, la corsa allo spazio e l’energia nucleare rappresentarono spinte decisive, poiché richiesero enormi capacità computazionali e innovazioni nell’uso delle risorse energetiche.

Un esempio pionieristico è rappresentato dall’ENIAC, costruito nel 1946. Utilizzava un sistema numerico digitale ed era dotato di una memoria capace di contenere 20 numeri da 10 cifre ciascuno. L’input veniva fornito tramite schede perforate e la macchina era in grado di eseguire operazioni come addizione, sottrazione, moltiplicazione, divisione e calcolo della radice quadrata.

In Italia, un altro esempio importante fu l’elaboratore ELEA 6001, un sistema elettronico aritmetico innovativo per l’epoca. La Olivetti, con la serie 9000, anticipò la produzione in serie di macchine da calcolo programmabili, che utilizzavano bobine a nastro per l’istruzione. Questo periodo segnò il massimo livello di ricerca e riconoscimento internazionale per l’Italia nel campo informatico. Tuttavia, a causa di una visione politica miope e di una mancanza di supporto, questi progetti non riuscirono a proseguire autonomamente, finendo per essere assorbiti da aziende americane.

Nel 1969, un’ulteriore grande innovazione fu il microprocessore, di cui Federico Faggin fu uno dei progettisti principali presso Intel. Egli fu anche il fondatore di Zilog e ideatore del microprocessore Z80, uno dei modelli più diffusi e utilizzati ancora oggi. Questo componente rivoluzionò la tecnologia informatica, permettendo di creare macchine più piccole, meno costose e più potenti.

All’inizio, i computer erano molto costosi e funzionavano come unità centrali condivise da più utenti contemporaneamente, collegati tramite lunghi cavi a terminali passivi di input e output. Successivamente, semplificando i processi e riducendo i costi, nacque il primo Personal Computer, destinato all’uso individuale e capace di svolgere un compito alla volta.

Con la diffusione dei PC, divenne necessaria la loro connettività tramite reti locali (LAN) e reti geografiche (WAN). In questo contesto nacquero anche Internet e il World Wide Web (WWW). Furono gli Stati Uniti, durante la corsa allo spazio, a sviluppare un sistema di comunicazione tra calcolatori che permettesse, in caso di interruzione del collegamento, di mantenere comunque attiva la comunicazione. Questa rete decentralizzata assicurava l’operatività continua, anche in condizioni di guasto.

\section{La quarta rivoluzione industriale}
La quarta rivoluzione industriale, in corso oggi, è caratterizzata da una diffusione capillare di computer interconnessi attraverso reti avanzate, da una grande efficienza energetica e da dimensioni sempre più ridotte delle macchine. Questo progresso è stato reso possibile grazie ai semiconduttori, che hanno permesso la creazione di dispositivi portatili come cellulari e laptop: strumenti a basso consumo energetico, con elevata potenza di elaborazione e grande portabilità. Tutto ciò ha cambiato radicalmente il modo in cui interagiamo con la tecnologia nella vita quotidiana.

Nuove tecnologie come il \emph{cloud} e l’\emph{Internet of Things (IoT)} stanno continuando a trasformare il nostro presente, creando scenari in cui la connessione e la condivisione dei dati sono alla base di nuovi modelli economici e sociali.
\section{Innovazione e Imprenditorialità}

Esistono diversi tipi di innovazione: di prodotto, di servizio, di processo, di business model e organizzativa. Le innovazioni possono essere incrementali o radicali.

L’imprenditore è colui che detiene fattori produttivi (capitali, mezzi di produzione, forza lavoro e materie prime, sotto forma di imprese), attraverso i quali, assieme ad investimenti, contribuisce a sviluppare nuovi prodotti, nuovi mercati o nuovi mezzi di produzione stimolando la creazione di una nuova ricchezza e valore sotto forma di beni e servizi utili alla collettività/società.

In Italia l’imprenditore viene definito dall’art. 2082 del Codice civile come:
\begin{quote}
    ``Chi esercita professionalmente un’attività economica organizzata al fine della produzione o dello scambio di beni o di servizi.''
\end{quote}

I motivi dell’imprenditorialità sono molteplici:
\begin{itemize}
    \item Desiderio di creare il proprio ``regno'' - aspetto umano.
    \item Desiderio di guadagno.
    \item Gusto di ``creare''.
    \item Challenge.
    \item Affermazione sociale.
\end{itemize}

L’imprenditore si impegna continuamente per realizzare ciò che rende ``unica'' la sua impresa, per combattere e battere la concorrenza e per incrementare il profitto.

Si utilizzano diverse modalità per cercare di distinguere la propria impresa:
\begin{itemize}
    \item I sei cappelli (bianco, giallo, verde, rosso, nero, blu).
    \item Lateral thinking, sviluppata da Edward de Bono.
    \item Vertical thinking.
\end{itemize}

Il valore è la somma che i compratori sono disposti a pagare per ottenere ciò che viene proposto loro da un'azienda, che identifica quell’ammontare con il nome di \textit{ricavo}.

Per incrementare il valore è necessario un vantaggio competitivo, che permette di essere in grado di praticare un prezzo più alto dei concorrenti e/o operare a costi minori dei concorrenti.

Per praticare dei prezzi più alti dei concorrenti è necessario aumentare la \textit{W.T.P. (willingness to pay)} tramite un’offerta unica e/o percepibile, creando valore per i clienti. Per operare a costi minori dei rivali è possibile diminuire i costi operativi e/o usare il capitale con efficienza per innovare, migliorare e gestire le attività.

Le attività sono funzioni economiche delimitate o processi, come gli approvvigionamenti, la gestione delle forze di vendita, lo sviluppo dei prodotti e la loro consegna ai clienti. Sono costituite normalmente da un insieme di persone, da tecnologia, attività fisse, capitale di esercizio e vari tipi di informazioni.

La sequenza delle attività eseguite da un’impresa per progettare, vendere e supportare i propri prodotti è chiamata \textit{catena del valore}. Ogni catena del valore è parte di un \textit{sistema del valore}.
